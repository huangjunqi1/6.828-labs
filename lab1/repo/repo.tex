\documentclass[UTF8,11pt]{ctexart}
\usepackage{graphicx}
\usepackage{enumerate}
\usepackage{color}
\usepackage{geometry} 
\usepackage{listings}
\usepackage{xcolor}
\usepackage[colorlinks,linkcolor=blue]{hyperref} 
\hypersetup{colorlinks=true,linkcolor=blue}
\definecolor{mygreen}{rgb}{0,0.6,0}
\definecolor{mygray}{rgb}{0.5,0.5,0.5}
\definecolor{mymauve}{rgb}{0.58,0,0.82}
\lstset{ %
    backgroundcolor=\color{white},      % choose the background color
    columns=flexible,
    breaklines=true,
    tabsize=4,
    commentstyle=\color{mygreen},  % comment style
    keywordstyle=\color{blue},     % keyword style
    stringstyle=\color{mymauve}\ttfamily,  % string literal style
    frame=none,
    rulesepcolor=\color{red!20!green!20!blue!20},
    language = c++,
    frame=single
% identifierstyle=\color{red},
}


\pagestyle{plain}
\geometry{left=2.2cm,right=2.3cm,top=1.5cm,bottom=1.5cm}
\title{Report for the 1st lab}
\author{姓名:黄骏齐mat={\Large\bfseries}]{section}
\and 学号:2100012956}
\date{2023年10月7日}

\CTEXsetup[for
\begin{document}
\maketitle
\section*{exercise2}
    Set register \verb|%ss| and \verb|%esp|(\verb|%esp| to \verb|0x7000|).

    Get some data from disk using \verb|in| and \verb|out|.

    Set PE in \verb|%cr0| to move in the protected mode and jump.
\section*{exercise3}

After \verb|%cr0| is reset, the computer enter 32-bit mode, through setting the PE symble in \verb|%cr0|.

The last instruction which boot loader executed is \verb|call *0x10018|

The first instruction of kernal is that at \verb|0x0010000c|, which is \verb|movw $0x1234, 0x472|.
\section*{exercise4}
\begin{lstlisting}
1: a = 000000000062FDC0, b = 0000000000BF1410, c = 0000000000000001
//a and b point to random address,c is a NULL pointer.
2: a[0] = 200, a[1] = 101, a[2] = 102, a[3] = 103
//a[x] == *(a+x)
3: a[0] = 200, a[1] = 300, a[2] = 301, a[3] = 302
//c=a, so 3[c]=c[3]=a[3]
4: a[0] = 200, a[1] = 400, a[2] = 301, a[3] = 302
//c=c+1 == c=(int*)((int c)+4)=> c points to a[1]
5: a[0] = 200, a[1] = 128144, a[2] = 256, a[3] = 302
//c points to ((char*)a + 1), and modify two integer.
6: a = 000000000062FDC0, b = 000000000062FDC4, c = 000000000062FDC1
//(int *) a + 1 = (int*)((int a) + 4); (int*)((char*)a+1) =  (int*)((int a) + 1) ;
\end{lstlisting}
\newpage
\section*{exercise5}

I modify \verb|0x7C00| to \verb|0x7C2D|, and get the result below.

\includegraphics*[scale=0.65]{ex5.png}

\section*{exercise6}

When entering bootloader, the few words at \verb|0x00100000| are all 0.

After loading the kernal, the few words at \verb|0x00100000| are not all 0.

The reason is bootloader load the kernal to \verb|0x00100000|.
\section*{exercise7}

Afer this instruction, both \verb|0x00100000| and \verb|0xf0100000| point to the same pysical address, and after comment out this instruction, the \verb|jmp| lead to crash.
\section*{exercise8}
Fill the code just like \verb|%h|.

\begin{lstlisting}
//in lib/printfmt.c
case 'o':
    // Replace this with your code.
    num = getuint(&ap, lflag);
    base = 8;
    goto number;
\end{lstlisting}
\subsection*{1}
\verb|console.c| export the function \verb|cputchar()| for \verb|printf.c|'s function \verb|putch|, which is used to print a single character. In the function \verb|vcprintf| in \verb|printf.c|, push \verb|putch| as an argument to the function \verb|vprintfmt|. 
\subsection*{2}
When the screen is full, move out the first line, move up the other lines to insert the new line.
\subsection*{3}
\verb|fmt| points to the char \verb|x|(the 1st char of the string \verb|x %d, y %x, z %d\n|),\verb|ap| points to the first argument (value \verb|x|)

\begin{enumerate}
    \item \verb|vcprintf (fmt=0xf0101937 "x %d, y %x, z %d\n", ap=0xf010ffd4 "\001")|
    \item \verb|cons_putc(c=120)|
    \item \verb|cons_putc(c=32)|
    \item \verb|va_arg| ap$\to$x $\Rightarrow$ ap$\to$y
    \item \verb|cons_putc (c=49)|
    \item \verb|cons_putc (c=44)|
    \item \verb|cons_putc (c=32)|
    \item \verb|cons_putc (c=121)|
    \item \verb|cons_putc (c=32)|
    \item \verb|va_arg| ap$\to$y $\Rightarrow$ ap-$\to$z
    \item \verb|cons_putc (c=51)|
    \item \verb|cons_putc (c=44)|
    \item \verb|cons_putc (c=32)|
    \item \verb|cons_putc (c=122)|
    \item \verb|cons_putc (c=32)|
    \item \verb|va_arg| ap$\to$z $\Rightarrow$ ap$\to$the address next z in the stack.
    \item \verb|cons_putc (c=52)|
    \item \verb|cons_putc (c=10)|
\end{enumerate}
\subsection*{4}
''He110 World''

Modify i to 0x726c6400.

No need to change 57616.
\subsection*{5}

the value at the address next \verb|&x| in the stack.

The \verb|ap| will point to the value next \verb|&x| in the stack and print it to the screen as an \verb|int|.

\subsection*{6}

I use another stack(manually) to offset the effect of reversal.
\section*{exercise9}
There is a instrction in \verb|entry.S| that \verb|movl	$(bootstacktop),%esp| to initialize the stack.

The virtual address \verb|0xf0110000| is the end of the stack.

\verb|.data| reserves the space of the stack.
\section*{exercise10}

\begin{enumerate}
    \item push the arguments reversely;
    \item push the return address(address that the next instrction of \verb|call|);
    \item push the \verb|%ebp| register of the last function;
    \item save \verb|%esp| in the \verb|%ebp| register;
    \item push callee registers to the stack;
    \item push some local values.
\end{enumerate}

Every execution of \verb|test_backtrace| will make \verb|%esp| move 32 byte (8 32-bit words.)

They're \verb|%eip|,\verb|%ebp|,\verb|%ebx|,and 5 arguments.
\subsection*{exercise11,12}
The type is \verb|N_SLINE|

\begin{lstlisting}
//in kern\kdebug.c
	stab_binsearch(stabs,&lline,&rline,N_SLINE,addr);
	if (lline>rline) return -1;
	info->eip_line = rline-lfile;
\end{lstlisting}


从汇编可以看出,进入函数时,先执行\verb|push %ebp|再\verb|mov %esp %ebp|,因此:

\verb|*ebp| saves the \verb|%ebp| this function.

\verb|*(ebp+1)| saves the \verb|%ebp| last function.

\verb|*(ebp+2)| to \verb|*(ebp+6)| is the first 5 arguments.
\newpage
\begin{lstlisting}
//in kern\monitor.c
int
mon_backtrace(int argc, char **argv, struct Trapframe *tf)
{
	cprintf("Stack backtrace:\n");
	int* ebp = ((int*)read_ebp());
	while (ebp != 0)
	{
		int eip = *(ebp+1);
		cprintf("  ebp %08x  eip %08x  args %08x %08x %08x %08x %08x\n",ebp+1,eip,*(ebp+2),*(ebp+3),*(ebp+4),*(ebp+5),*(ebp+6));
		struct Eipdebuginfo info;
		int tmp = debuginfo_eip(eip, &info);
		cprintf("         %s:%d: ",info.eip_file,info.eip_line);
		for (int i=0;i<info.eip_fn_namelen;i++) //info.eip_fn_namelen saves the length of the function
			cprintf("%c",info.eip_fn_name[i]);
		cprintf("+%d\n",info.eip_fn_narg);
		ebp = (int*)(*(ebp));
	}
	return 0;
}
\end{lstlisting}

\end{document}

